% "Shell Programming" talk
% Copyright (C) 2008  Chris Lamb <chris@chris-lamb.co.uk
%
% Based on a template (C) 2007, 2008 Daniel Watkins <D.M.Watkins@warwick.ac.uk>
%                     (C) 2007, 2008 Chris Lamb <chris@chris-lamb.co.uk>
%
%  This program is free software; you can redistribute it and/or modify
%  it under the terms of the GNU General Public License as published by
%  the Free Software Foundation; either version 3 of the License, or
%  (at your option) any later version.
%
%  This program is distributed in the hope that it will be useful,
%  but WITHOUT ANY WARRANTY; without even the implied warranty of
%  MERCHANTABILITY or FITNESS FOR A PARTICULAR PURPOSE.  See the
%  GNU General Public License for more details.
%
%  You should have received a copy of the GNU General Public License
%  along with this program; if not, write to the Free Software
%  Foundation, Inc., 51 Franklin St, Fifth Floor, Boston, MA  02110-1301  USA

\documentclass{beamer}

\usepackage{beamerthemesplit}
\usetheme{Warsaw}

\usepackage{graphicx}
\usepackage{url} 

\usepackage{listings} 
\lstset{basicstyle=\ttfamily}


\title{Shell Programming}
\author[Chris Lamb, WUGLUG]{Chris Lamb\\Warwick University GNU/Linux User Group}
\date{20th February 2008
\newline
\newline
\tiny{The \LaTeX{} source code for this presentation is licensed under version 3 of the GNU General Public License.}}

\begin{document}

\frame{\titlepage}

%

\section{Shell scripts}
\subsection{Shell programming != Bash programming}
    % Bash is buggy and slow. (Planet Debian)

\subsection{Why not write in X?}
    % C
    % Python
    % Haskell
    % Ruby
    % Perl

%

\section{Writing robust scripts}
\subsection{set -e and set -u}
    % set -e
        % Example
        % Caught in pipes
        % When a command might fail
        % In subshells (!)
    % set -u
        % Example
        % Detecting unset variables under set -u
\subsection{Traps and locks}
    % Cleanup
\subsection{Misc}
    % Spaces in filenames
    % ${FOO} vs $FOO
    % Set PATH
    % IFS
\subsection{Bashisms}
    % Variable expansions
    % bashisms-superset.png
    % bashisms-evil.png
        % Syntax that is valid Bash AND Posix shell but has different semantics

\subsection{Robust Makefiles}
    % Psuedo set -e:
        % Everyone knows that the following snippet will fail on the third line:
        %   all:
        %       echo "Hello, from the '$@' target"
        %       false
        % But what about:
        %   all:
        %       echo "foo"; false; echo "bar"
        % Solution is to add 'set -e' at the beginning of multi-line.
    % No set -u. Not yours.
    % No $(cmd), just ugly ``

%

\section{Toolbox}
\subsection{awk}
    % Useful for 'record' oriented data
    %   getent passwd
    %
    %  awk [-F "field-sep"] '
    %     [ BEGIN { expr } ]
    %     [ END { expr } ]
    %     [[<condition>] { expr }]
    %
    % Use my example.
    % Replicate existing commands
\subsection{sed}
    % Hacky stuff
    % sed -i <expr> ( <expr> ) *
    % sed -n 's/search/replace/p'
\subsection{echo}
    % Don't rely on anything there
\subsection{Processing arguments}
    % while / #? / shift
\subsection{Local variables}
    % local -a
    % Scoping defined
\subsection{find and xargs}
    % find -exec looks nice, but don't forkbomb
    % Use find -print0 | xargs -0
\subsection{Debugging}
    % Black art
    % /bin/sh -n (noexec)
    % set -x (turn off with set +x)
    % Checkbashisms

%

\frame {
    \frametitle{Thanks!}
    WUGLUG contact information:
    \begin{itemize}
        \item Website: \url{http://www.wuglug.org.uk}
        \item IRC: {\tt \#wuglug} on {\tt irc.uwcs.co.uk:6667}
        \item Mailing list: \url{https://mailman.warwickcompsoc.co.uk/listinfo/wuglug}
    \end{itemize}
}

\end{document}
